\documentclass[transparent]{beamer}
\usepackage[frenchb]{babel}
\usepackage[T1]{fontenc}
\usepackage[utf8]{inputenc}
\usepackage{ulem}
\usepackage{media9}
\usepackage{hyperref}
\usepackage{listings}

\usetheme{Warsaw}


\title{PFE \\ }
\subtitle{Drone autonome}
\author{Benjamin  Ravalijaona - Nicolas DURAN - Rodolphe GUITTENY}
\institute{SCIA 2015 \\ EPITA}
\date{17 mai 2014}

\begin{document}

\begin{frame}
	\titlepage
\end{frame}


\begin{frame}
	\frametitle{Spécifications Fonctionnelles}
	\begin{block}{Le projet}
			\begin{itemize}
				\item Développement d’un logiciel de contrôle de drone para-moteur équipé d’un téléphone 3G et d’un appareil vidéo GoPro.
			\end{itemize}
	\end{block}
\end{frame}

\begin{frame}
	\frametitle{Spécifications Fonctionnelles}
	\begin{block}{Objectifs :}
			\begin{itemize}
				\item Détection d’une voiture à partir d’image extraite par la GoPro fixée sur le drone.
				\item Suivi de la voiture par le drone (Tracking).
				\item Autonomie de l'appareil sur l'ensemble du trajet.
			\end{itemize}
	\end{block}
\end{frame}

\begin{frame}
	\frametitle{Spécifications Fonctionnelles}
	\begin{block}{Remarque:}
			\begin{itemize}
				\item Pour des raisons de complexité, le décollage et l’atterrissage du drone ne seront pris en charge par le logiciel.
				\item Ces opérations seront effectuées par l'intermediaire de l'ordinateur, manuellement.
			\end{itemize}
	\end{block}
\end{frame}

\begin{frame}
	\frametitle{Spécifications non Fonctionnelles}
	\begin{block}{Caractéristiques du téléphone:}
			\begin{itemize}
				\item RAM
				\item CPU
				\item version Android
				\item Rootage téléphone ?
			\end{itemize}
	\end{block}

	\begin{block}{Caractéristiques du drone:}
			\begin{itemize}
				\item Model :
			\end{itemize}
	\end{block}

	\begin{block}{Caractéristiques de la GoPro:}
			\begin{itemize}
				\item Communication Wifi
				\item Image XXMP
			\end{itemize}
	\end{block}
\end{frame}

\begin{frame}
	\frametitle{Spécifications techniques générales}
	\begin{block}{Choix du Téléphone:}
			\begin{itemize}
				\item Peu couteux proportionnellement au nombre de capteurs inclus.
				\item Facile d'utilisation.
				\item Fourni par Intel.
			\end{itemize}
	\end{block}
	\begin{block}{Ajout d'un appareil vidéo GoPro:}
			\begin{itemize}
				\item Bonne qualitée vidéo permettant des traitements de l'image.
				\item Possibilité de communicaton en wifi.
				\item Requiert un investissement.
			\end{itemize}
	\end{block}
\end{frame}

\begin{frame}
	\frametitle{Spécifications techniques générales}
	\begin{block}{Choix du Drone:}
			\begin{itemize}
				\item Autonomie.
				\item Possibilité de porter une certaine charge (ajout de GoPro, et potentiellement de batterie).
				\item Stabilité.
				\item Fourni par Intel.
			\end{itemize}
	\end{block}
\end{frame}

\begin{frame}
\frametitle{Spécifications techniques générales}
	\begin{block}{Choix du réseau 3G:}
			\begin{itemize}
				\item Communication entre ordinateur des drone pour les controles manuels (décollage/atterrissage).
				\item Large couverture du réseau 3G
			\end{itemize}
	\end{block}
	\begin{block}{Choix du réseau Wifi:}
			\begin{itemize}
				\item Communication entre téléphone et appareil video GoPro.
				\item Bande passante importante.
				\item Faible latence.
			\end{itemize}
	\end{block}
\end{frame}

\end{document}
